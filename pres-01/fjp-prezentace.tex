\documentclass[xcolor=dvipsnames]{beamer}
%\usepackage[utf8]{inputenc}
%\usepackage{xcolor}
\usepackage{graphicx}
\usepackage{tikz}
\usetikzlibrary{arrows,shapes}
\usepackage{caption}
%\usepackage[utf8]{inputenc}
\usepackage[czech]{babel}
%\usepackage[utf8]{vietnam}
\usepackage{pdfpages}
\usepackage{color}
\usepackage{booktabs}

%%%%%%%%%%%%%%%%%%%%%%%%%%%%%%%%%%%%%%%%%%%%%%%%%%%%%

%%%%%%%%%%%%%%%%%%%%%%%%%%%%%%%%%%%%%%%%%%%%%%%%%%%%%
%\usepackage{lipsum}


%%%%%%%%%%%%%%%%%%%%%%%%%%%%%%%%%%%%%%%%%%%%%%%%%%%%%

\usepackage{pgf}
\usepackage{etex}
\usepackage{tikz,pgfplots}

\tikzstyle{every picture}+=[remember picture]
% By default all math in TikZ nodes are set in inline mode. Change this to
% displaystyle so that we don't get small fractions.
\everymath{\displaystyle}


\usetheme{Antibes}
%\usetheme{Madrid}
%\usecolortheme[named=Maroon]{structure}
\usecolortheme{crane}
\usefonttheme{professionalfonts}
\useoutertheme{infolines}
\useinnertheme{circles}

\newtheorem*{bem}{Bemerkung}

\usepackage{tikz}


%%%%%%%%%%%%%%%%%%%%%%%%%%%%%%%%%%%%%%%%%%%%%%%%%



%%%%%%%%%%%%%%%%%%%%%%%%%%%%%%%%%%%%%%%%%%%%%%%%%
%\usepackage{listings}
\usepackage{color}

\definecolor{dkgreen}{rgb}{1,0.6,0}
\definecolor{gray}{rgb}{1,1,0}
\definecolor{mauve}{rgb}{0.58,0,0.82}

%%%%%%%%%%%%%%%%%%%%%%%%%%%%%%%%%%%%%%%%%%%%%%%%%

\title[Západočeská univerzita v Plzni]{\includegraphics[width=\textwidth/4]{img/logo.png}}
\author[Jindřiška Reismüllerová \& Stanislav Král]{ \textbf{FJP -- Návrh jazyka C - -}}
\institute{}
%\logo{\includegraphics[height=1.5cm]{img/KIV\_ram\_cerna.pdf}}

\logo{\pgfimage[height=0.5cm]{img/kiv-logo.pdf}}

\setbeamercolor{title}{parent=structure}

\begin{document}

\begin{frame}
  \titlepage
\end{frame}

\begin{frame}
\frametitle{\textbf{Zadání}}
	\begin{itemize}
    		\item Zvolené 4. možné zadání - Tvorba vlastního překladače
    		\item Překlad podmnožiny     jazyka C
    
  	\end{itemize}
\end{frame}


\begin{frame}
\frametitle{\textbf{Popis jazyka}}
	\begin{itemize}
    		\item Silně staticky typovaný
    		\item Datové typy - \texttt{int}, \texttt{string}, \texttt{bool}, \texttt{array}, \texttt{struct}
    		\item Binární operátory - \uv{+-*/}
    		\item Operátory pro porovnání - \uv{$<,  >,   ==,  <=,  >= ,  !=$}
    		\item Logické operátory - \uv{$||$, \&\&}
    		\item Řídící struktury - \texttt{for}, \texttt{while}, \texttt{if} (\texttt{else})
  	\end{itemize}
\end{frame}

\begin{frame}
\frametitle{\textbf{Další rozdíly oproti C}}
	\begin{itemize}
    		\item Nemožnost použití operátorů ++/- -
    		\item Řídící proměnná musí být deklarovaná před cyklem
    		\item Nepřítomnost ukazatelů
  	\end{itemize}
\end{frame}

\begin{frame}
\frametitle{\textbf{Technologie}}
	\begin{itemize}
    		\item Jazyk C/C++
    		\item Flex - lexikální analýza
    		\item Bison - syntaktická analýza
    		\item Cílová platforma -  PL/0
  	\end{itemize}
\end{frame}

\begin{frame}[fragile]
\frametitle{\textbf{Gramatika - úvod}}
    \begin{itemize}
        \item V tuto chvíli 18 neterminálních symbolů a 49 přepisovacích pravidel
    \end{itemize}
    \begin{figure}
        \centering
        \begin{verbatim}
    program ->      promenna program |
                    funkce program |
                    e

    promenna ->     deklarace ";"|
                    deklarace "=" hodnota ";"|
                    "const" deklarace "=" hodnota ";"|
                    struktura ";"

    deklarace ->    dat_typ identifikator |
                    dat_typ identifikator "[" cislo "]"
        \end{verbatim}
    \end{figure}
\end{frame}

\begin{frame}[fragile]
\frametitle{\textbf{Gramatika - struktury}}
    \begin{itemize}
        \item Již počítá s podporou polí a definice struktur
        \item Středník za každým výrazem či deklarací proměnné
    \end{itemize}
    \begin{figure}
        \centering
        \begin{verbatim} 
    multi_deklarace ->  deklarace ";" | 
                        deklarace ";" multi_deklarace
    
    struktura       ->  "struct" identifikator "{" 
                            multi_deklarace 
                        "}" 
        \end{verbatim}
    \end{figure}
\end{frame}

\begin{frame}[fragile]
\frametitle{\textbf{Gramatika - funkce}}
    \begin{figure}
        \centering
        \begin{verbatim}
    funkce ->   deklarace "(" parametry ")" blok

    parametry -> deklarace |
                 deklarace "," parametry |
                 e
    
    blok ->     "{" prikazy "}" | prikaz
    
    prikazy ->  prikaz |
                prikaz prikazy
        \end{verbatim}
    \end{figure}
\end{frame}

\begin{frame}[fragile]
\frametitle{\textbf{Gramatika - příkazy a výrazy}}
    \begin{figure}
        \centering
        \begin{verbatim}
    prikaz -> promenna |
              podminka |
              vyraz ";"|
              cyklus |
              "return" vyraz ";"
              
    vyraz ->  identifikator "=" hodnota |
              identifikator "[" cislo "]" "=" hodnota |
              volani_fce |
              logicky_vyraz |
              hodnota
              
    vyrazy -> vyraz |
              vyraz "," vyrazy |
              e
        \end{verbatim}
    \end{figure}
\end{frame}

\begin{frame}[fragile]
\frametitle{\textbf{Gramatika - řízení programu}}
    \begin{figure}
        \centering
        \begin{verbatim}
    podminka -> "if" "(" vyraz ")" blok |
                "if" "(" vyraz ")" blok "else" blok
    
    cyklus ->   "while" "(" vyraz ")" blok |
                "for" "(" vyraz ";" vyraz ";" vyraz ")" blok
                
    volani_fce -> identifikator "(" vyrazy ")"
        \end{verbatim}
    \end{figure}
\end{frame}

\begin{frame}[fragile]
\frametitle{\textbf{Gramatika - hodnoty a logické výrazy}}
    \begin{figure}
        \centering
        \fontsize{10pt}{12pt}\selectfont
        \begin{verbatim}
    hodnota ->       cislo |
                     string |
                     identifikator |
                     matematika |
                     volani_fce |
                     identifikator "[" cislo "]"

    matematika ->    hodnota bin_operator hodnota

    logicky_vyraz -> hodnota |
                     hodnota porovnani hodnota |
                     "!" logicky_vyraz |
                     "(" logicky_vyraz ")" |
                     logicky_vyraz log_oper logicky_vyraz
        \end{verbatim}
    \end{figure}
\end{frame}

\begin{frame}
\frametitle{\textbf{Závěr}}
	\begin{itemize}
    		\item Gramatika se ještě změní s výběrem volitelných rozšíření jazyka
    		\item Volba C/C++ pro tvorbu překladače ještě není finální
  	\end{itemize}
\end{frame}

%*********************************************************************************************

%**********************************************************************************************
%\begin{frame}\frametitle{}
%  \centering
%    \includemedia[
%  width=0.6\linewidth,height=0.45\linewidth,
%  activate=pageopen,
%  flashvars={
%    modestbranding=1 % no YT logo in control bar
%   &autohide=1       % controlbar autohide
%   &showinfo=0       % no title and other info before start
%  }
%]{}{http://www.youtube.com/v/dISaXUlilkU?rel=0}   % Flash file 
      
%\end{frame}
%%%%%%%%%%%%%%%%%%%%%%%%%%%%%%%%%%%%%%%%%%%%%%%%%%%
\end{document}

