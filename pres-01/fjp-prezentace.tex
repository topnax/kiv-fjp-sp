\documentclass[xcolor=dvipsnames]{beamer}
%\usepackage[utf8]{inputenc}
%\usepackage{xcolor}
\usepackage{graphicx}
\usepackage{tikz}
\usetikzlibrary{arrows,shapes}
\usepackage{caption}
%\usepackage[utf8]{inputenc}
\usepackage[czech]{babel}
%\usepackage[utf8]{vietnam}
\usepackage{pdfpages}
\usepackage{color}
\usepackage{booktabs}

%%%%%%%%%%%%%%%%%%%%%%%%%%%%%%%%%%%%%%%%%%%%%%%%%%%%%

%%%%%%%%%%%%%%%%%%%%%%%%%%%%%%%%%%%%%%%%%%%%%%%%%%%%%
%\usepackage{lipsum}


%%%%%%%%%%%%%%%%%%%%%%%%%%%%%%%%%%%%%%%%%%%%%%%%%%%%%

\usepackage{pgf}
\usepackage{etex}
\usepackage{tikz,pgfplots}

\tikzstyle{every picture}+=[remember picture]
% By default all math in TikZ nodes are set in inline mode. Change this to
% displaystyle so that we don't get small fractions.
\everymath{\displaystyle}


\usetheme{Antibes}
%\usetheme{Madrid}
%\usecolortheme[named=Maroon]{structure}
\usecolortheme{crane}
\usefonttheme{professionalfonts}
\useoutertheme{infolines}
\useinnertheme{circles}

\newtheorem*{bem}{Bemerkung}

\usepackage{tikz}


%%%%%%%%%%%%%%%%%%%%%%%%%%%%%%%%%%%%%%%%%%%%%%%%%



%%%%%%%%%%%%%%%%%%%%%%%%%%%%%%%%%%%%%%%%%%%%%%%%%
%\usepackage{listings}
\usepackage{color}

\definecolor{dkgreen}{rgb}{1,0.6,0}
\definecolor{gray}{rgb}{1,1,0}
\definecolor{mauve}{rgb}{0.58,0,0.82}

%%%%%%%%%%%%%%%%%%%%%%%%%%%%%%%%%%%%%%%%%%%%%%%%%

\title[Západočeská univerzita v Plzni]{\includegraphics[width=\textwidth/4]{img/logo.png}}
\author[Jindřiška Reismüllerová \& Stanislav Král]{ \textbf{FJP -- Návrh jazyka C - -}}
\institute{}
%\logo{\includegraphics[height=1.5cm]{img/KIV\_ram\_cerna.pdf}}

\logo{\pgfimage[height=0.5cm]{img/kiv-logo.pdf}}

\setbeamercolor{title}{parent=structure}

\begin{document}

\begin{frame}
  \titlepage
\end{frame}

\begin{frame}
\frametitle{\textbf{Zadání}}
	\begin{itemize}
    		\item 4. Tvorba vlastního překladače
    		\item Podmnožina jazyka C
    
  	\end{itemize}
\end{frame}


\begin{frame}
\frametitle{\textbf{Popis jazyka}}
	\begin{itemize}
    		\item Silně staticky typovaný
    		\item Datové typy - int, string, bool
    		\item Binární operátory - \uv{+-*/}
    		\item Operátory pro porovnání - \uv{$<,  >,   ==,  <=,  >= ,  !=$}
    		\item Logické operátory - \uv{$||$, \&\&}
    		\item Řídící struktury - for, while, if (else)
  	\end{itemize}
\end{frame}

\begin{frame}
\frametitle{\textbf{Další rozdíly oproti C}}
	\begin{itemize}
    		\item Nemožnost ++/- -
    		\item Řídící proměnná musí být deklarovaná před cyklem
    		\item Nepřítomnost ukazatelů
  	\end{itemize}
\end{frame}

\begin{frame}
\frametitle{\textbf{Technologie}}
	\begin{itemize}
    		\item Jazyk C/C++
    		\item Flex - lexikální analýza
    		\item Bison - syntaktická analýza
    		\item Cílová platforma -  PL/0
  	\end{itemize}
\end{frame}

\begin{frame}
\frametitle{\textbf{Gramatika}}
	
\end{frame}
            
  
   
%*********************************************************************************************

%**********************************************************************************************
%\begin{frame}\frametitle{}
%  \centering
%    \includemedia[
%  width=0.6\linewidth,height=0.45\linewidth,
%  activate=pageopen,
%  flashvars={
%    modestbranding=1 % no YT logo in control bar
%   &autohide=1       % controlbar autohide
%   &showinfo=0       % no title and other info before start
%  }
%]{}{http://www.youtube.com/v/dISaXUlilkU?rel=0}   % Flash file 
      
%\end{frame}
%%%%%%%%%%%%%%%%%%%%%%%%%%%%%%%%%%%%%%%%%%%%%%%%%%%
\end{document}
